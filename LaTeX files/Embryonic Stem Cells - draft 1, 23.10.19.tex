\documentclass[11pt, a4]{article}
\usepackage[a4paper, top=1in, bottom=1in, left=1in, right=1in]{geometry}
\usepackage{color} 
\usepackage[pdftex]{hyperref}
\usepackage[export]{adjustbox}
\usepackage{float}
\usepackage{array}
\usepackage[activate={true,nocompatibility},final,tracking=true,kerning=true,spacing=true,factor=1100,stretch=10,shrink=10]{microtype}
\usepackage{tikz}
\usepackage{graphicx}
\usepackage{wrapfig}
\usepackage{pgfplots}
\usepackage{multirow}
\usepackage{longtable}
\usepackage{array}
\usepackage[justification=centering]{caption} 
\usepackage{titlesec}

\hypersetup{colorlinks=true, linktoc=all, linkcolor=black,}
\restylefloat{table}

\title{\vspace{90mm}Embryonic Stem Cells}
\author{Adithya Narayanan}
\date{6 November, 2019}
\definecolor{pastelgreen}{rgb}{0.47, 0.87, 0.47}

\begin{document}
	\begin{titlepage}
		\maketitle
		\centering
			Word count: 1198
		\thispagestyle{empty}
	\end{titlepage}
	\newpage
	\tableofcontents
	\thispagestyle{empty}
	\newpage
	\setcounter{page}{1}
	\section{Human Embryonic Stem Cells (hESCs):}
	
		\begin{wrapfigure}{r}{4 cm}
			\centering
			\includegraphics[width=4 cm, trim=0cm 0cm 0cm 3cm]{How_embryonic_stem_cells_are_derived}
			\caption{\footnotesize National Institutes of Health. \textit{How Embryonic Stem Cells Are Derived}, 2006, stemcells.nih.gov/info/
				Regenerative\_Medicine/2006
				Chapter1.htm. Accessed 23 Oct. 2019.}
			\vspace{-55pt}
		\end{wrapfigure}
	
		Embryonic Stem Cells(ESCs), as the name may suggest, are derivations of the embryoblast inside the primordial embryo, during the blastocyst stage(see Fig.1), approximately 4-5 days post fertilization and prior to adherence to the uterus wall. Cultured and extracted during the pre-implantation stages of processes such as in vitro fertilization, hESCs exhibit 2 remarkable characteristics-pluripotence prior to maturity and a seeming lack of a hayfleck limit, exhibiting a great proliferative potential.
		
		\subsection{Pluripotence:}
			The embryoblast of a blastocyst stage embryo is undifferentiated, allowing for the ability to induce specialization under specific conditions and gain functions of specialized cells such as ``neural cells, cardiomyocytes, hematopoietic precursors [and] $\beta$-like cells''(Yu and Thomson). hESCs have the capacity to develop into all cells of the human body, with the exception of ``extra-embryonic tissues''(Nature). hESCs are special in this regard, as ``no other cell in the body has the natural ability to generate new cell types''(Mayo Clinic).
			
		\subsection{Proliferative potential:}
			When isolated from the embryo and grown in a Petri dish, hESCs remain the only cell in the body, other than cancer cells, with the ability to divide indefinitely and maintain characteristics such as their undifferentiated state. 
		
		\section{Acute Myeloid leukaemia (AML) in individuals over 60 years of age:}
			
			\begin{wrapfigure}{r}{5 cm}
				\centering
				\includegraphics[width=5 cm]{Myeloblast.jpg}
				\caption{\footnotesize National Cancer Institute. \textit{Blood Cell Development}, www.cancer.gov/types/leukemia
					/patient/adult-aml-treatment-pdq. Accessed 23 Oct. 2019.}
				\vspace{-20pt}
			\end{wrapfigure}
			
			AML is a cancer developing from the bone marrow(see Fig.3), which consists of multipotent blood SCs(see Fig.2) that  can ``grow into all 3 types of blood cells – red cells, white cells and platelets''(AAMDS), and naturally utilise 2 mechanisms to produce blood cells. In obligate-asymmetric-replication, the SC divides into a mother and daughter cell, where the mother is a replicate of the original SC and the daughter is a committed progenitor, that becomes the specialized blood cell. In stochastic-differentiation, the SC divides into 2 specialized cells which signals another SC to undergo mitosis and divide into 2 SCs.
				
			\bigbreak
				
			In AML, blood SCs develop into abnormal myeloblasts, known as leukaemia cells, during the 2 above mentioned processes, and crowd out the blood forming tissue in the bone marrow, affecting haemopoiesis. ``Eventually, a person will start to lack RBCs that carry oxygen, platelets that prevent easy bleeding, and WBCs that protect the body from diseases''(Nall).
				\newpage
		\section{Treatments for AML:}
			\begin{wrapfigure}{r}{4.5 cm}
				\centering
				\includegraphics[width=4.5 cm, trim=0cm 0cm 0cm 5cm]{Bone_anatomy.jpg}
				\caption{\footnotesize National Cancer Institute. \textit{Bone Anatomy}, www.cancer.gov/types/leukemia
					/patient/adult-aml-treatment-pdq. Accessed 23 Oct. 2019.}
				\vspace{-50pt}
			\end{wrapfigure}
			
			The typical treatment for AML is chemotherapy given in 3 phases(Induction, Consolidation and Maintenance\footnote{This stage is typically only undergone if the patient is diagnosed with \newline Acute Promyelocytic Leukaemia(APL)}), as preparation for a Stem Cell Transplant(SCT) that follows, which can be either allogeneic(see Fig.4) or autologous(see Fig.5).
			
			\bigbreak
				
			Other treatments include radiation therapy, used when the leukaemia cells spread outside the bone marrow, FLT3, BCL-2, hedgehog pathway and IDH inhibitors, that block proteins in cancer cell genes, Gemtuzumab ozogamicin(a monoclonal antibody), Arsenic-trioxide and all-trans-retinoic-acid(ATRA, a form of vitamin A).
				
			\subsection{Allogenic SCT:}
			
				\begin{wrapfigure}{r}{4 cm}
					\centering
					\includegraphics[width=4 cm]{Allogenic_SCT.jpg}
					\caption{\scriptsize Leukemia and Lymphoma society. \textit{Allogenic Stem Cell Transplantation}, www.lls.org/treatment/types-of-treatment/stem-cell-transplantation/allogeneic-stem-cell-transplantation. Accessed 23 Oct. 2019.}
					\vspace{-50pt}
				\end{wrapfigure}
				
				In an allogenic SCT, blood SCs or hESCs from a donor whose HLA type closely matches the recipient, usually a close relative or Matched Unrelated Donor(MUD), are transplanted. In a Non-myeloablative transplant(NMT), lower doses of chemotherapy are given, coupled with SCT using donor SCs that ``establish a new immune system, which see the leukaemia cells as foreign and attack them''(American Cancer Society) instead, creating a graft-versus-leukaemia effect. More recent clinical trials have also shown ESCs to be a valid option, by committing ESCs to multipotent blood SCs and utilising them during SCT.
				
				
			\subsection{Benefits of SCT over other treatment methods:}
				In typical chemotherapy, doses are limited as to prevent damage to the bone marrow. However, with SCT, higher doses can be given as destroyed bone marrow can be replaced with a blood or committed hESCT. SCT can also ``start a new supply of red cells, white cells and platelets''(Leukemia and Lymphoma society) and is the only viable long term cure for AML.			
			
			\subsection{Benefits of allogenic SCT over other SCT treatment methods:}
				
				Allogenic SCT has numerous benefits over other SCT treatment methods. The graft-versus-leukaemia effect, can recognise leukaemia cells as foreign and attempt to remove them. Allogenic SCT also completely avoids the chances of being contaminated with leukaemic cells, unlike autologous SCT where there are possibilities for leukaemic cells to be present even after purging. NMTs can also reduce chemotherapy doses, something not possible with autologous SCT, thus being more suited for the elderly.
				
			\subsection{Limits of SCT over other treatment methods:}
			\begin{wrapfigure}{r}{4 cm}
				\centering
				\includegraphics[width=4 cm, trim=0cm 0cm 0cm 1cm]{Autologous_SCT.jpg}
				\caption{\footnotesize Leukemia and Lymphoma society. \textit{Autologous Stem Cell Transplantation}, www.lls.org/treatment/types-of-treatment/stem-cell-transplantation/allogeneic-stem-cell-transplantation. Accessed 23 Oct. 2019.}
				\vspace{-70pt}
			\end{wrapfigure}	
			
				The main disadvantage with SCT is the need to perform chemotherapy prior. Chemotherapy has numerous side effects, such as hair loss, mouth sores, vomiting/nausea, diarrhoea/constipation, easy bleeding or bruising and an increased risk of infection, especially in the elderly. There is also the problem with chemotherapy not being effective, requiring other treatments such as radiation therapy in order to proceed with SCT.
				
			\subsection{Limits of allogenic SCT over other SCT treatment methods:}
				The major disadvantage with allogenic SCT, especially with hESCs, is the possibility of a graft-versus-host-disease(GVHD) occurring, where ``the patient’s immune system is taken over by that of the donor''(American Cancer Society). If significant enough, GVHD may be life-threatening and is not an issue present with autologous SCT. There is also the possibility that the patient cannot find a suitable donor, one of the more common issues.
				
	\section{World factor (ethical):}
	
		\bigbreak
		
		The largest problem with hESCs is the dilemma between relieving human suffering or valuing human life. When hESCs are extracted, the embryo is destroyed. The argument presented here, is that if let to develop, the embryo would become a human and hence destroying the embryo would be tantamount to murder, however the other argument, is that ``moral status is based on features and capacities none of which the embryo possesses [and] that the embryo is ‘simply a collection of cells’''(Chan). 
	
		\bigbreak
	
		However, the issue here is defining when an embryo gains the rights of a human being and at what point the extraction of hESCs would be	considered as harming/destroying a life, hence it is difficult to define when it is a human and when it is not. Some argue that the embryo has the rights of a human at the point of fertilization itself, while others believe that ``an early embryo that has not yet been implanted into the uterus does not have the psychological, emotional or physical properties that we associate with being a person''(Euro Stem Cell) and hence cannot be defined as such. The opposing party would also argue that if there are embryos no longer of use, such as in the scenario of in-vitro fertilization, the embryo may be used for the purpose of saving lives(when given consent), rather than letting them waste away.

		\bigbreak
	
		Some others also argue that the moral status of the embryo increases as it develops. However, on the other end of the spectrum, some argue that the embryo does not have a moral status at all, and is a mere piece of organic matter. In the end, one central argument remains - if given explicit consent by the donor, the fertilized embryo may be used for research purposes.
	
	\section{Evaluation:}
		With the multiple arguments presented, it is evidently hard to present a definitive judgement for this dilemma. It would be wrong to destroy an embryo that was intended to be used by a human, however, the criteria are not quite the same for an embryo that is intended to be thrown away. It would be seemingly okay to use it for research purposes, especially if given consent by the donor, as the embryo would die either way, when left alone or when used for research purposes and the excess embryos may as well be used to save lives, rather than being left to waste away.
		
		\bigbreak
		
		In the end however, although there are many arguments to be made, it is clear that further utilisation of hESCs must bring into question the morality and ethicality of destroying an embryo to save a life or recognise the embryo as a human with a right to live that must be taken into account.

\end{document}